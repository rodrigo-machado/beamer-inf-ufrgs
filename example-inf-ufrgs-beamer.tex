\documentclass[xcolor=dvipsnames,pdf,10pt]{beamer}
\usetheme{inf-ufrgs}

%%% Encoding and Fonts
\usepackage[utf8]{inputenc}
\usepackage[T1]{fontenc}
\usefonttheme{professionalfonts}
\usepackage{lmodern}
\usepackage[small]{eulervm}


%%% EASY ENUMERATIONS
\newcommand{\bi}{\begin{itemize}}
\newcommand{\ei}{\end{itemize}}
\newcommand{\be}{\begin{enumerate}}
\newcommand{\ee}{\end{enumerate}}
\newcommand{\bd}{\begin{description}}
\newcommand{\ed}{\end{description}}
\newcommand{\tm}{\item}

%%% THEOREMS, DEFINITIONS, ETC... 
\newcommand{\defn}{\textcolor{blue}{\textbf{\textrm{Definition:\ }}}}
\newcommand{\thm}{\textcolor{OliveGreen}{\textbf{\textrm{Theorem:\ }}}}
\newcommand{\lem}{\textcolor{OliveGreen!50!red}{\textbf{\textrm{Lemma:\ }}}}
\newcommand{\crlr}{\textcolor{green!50!orange!80!black}{\textbf{\textrm{Corolary:\ }}}}
\newcommand{\prf}{\textcolor{Red}{\textbf{\textrm{Proof:\ }}}}
\newcommand{\exmp}{\textcolor{WildStrawberry}{\textbf{\textrm{Example:\ }}}}
\newcommand{\rmk}{\textcolor{Purple}{\textbf{\textrm{Remark:\ }}}}
\newcommand{\excs}{\textcolor{Brown}{\textbf{\textrm{Exercise:\ }}}}
\newcommand{\qstn}{\textcolor{WildStrawberry!90!black}{\textbf{\textrm{Question:\ }}}}
\newcommand{\ansr}{\textcolor{violet}{\textbf{\textrm{Answer:\ }}}}
\newcommand{\notn}{\textcolor{gray!50!black}{\textbf{\textrm{Notation:\ }}}}


%%%%%%%%%%%%%%%%%%%%%%%%%%%%%%%%%%%%%%%%%%%%%%%%%%%%%%%%%%%%%%%%%%%%%%%%%%%%%%%%%%%%
\title     [Example presentation]      {Example presentation}
\subtitle                              {Addressing multiple topics}
\author    [Author]                    {Author\\ \texttt{author@inf.ufrgs.br}}
\institute                             {\inftitle} 
\date                                  {Some day in 2014}
%%% O texto entre [ ] vai aparecer na barra inferior...

\renewcommand{\contentstitle}{Outline of this talk}  
%%% Título usado nos slides de seção introduzidos automaticamente
%%%%%%%%%%%%%%%%%%%%%%%%%%%%%%%%%%%%%%%%%%%%%%%%%%%%%%%%%%%%%%%%%%%%%%%%%%%%%%%%%%%%

\usepackage{listings}
\lstset{language=Java,basicstyle=\ttfamily}

\begin{document}

\titlepageINF

\tableofcontentsINF

\section{About $e$} 

%----------------------------------------------------------------------------------%
\begin{frame}[allowframebreaks]{The constant $e$}

\ 

\defn 
\[e =   \lim_{n \to \infty} \left(1 + \frac{1}{n}\right)^n \]


\ 

\prf 
\be
\tm $\ln$ in both sides.
  \[ 1 =   \ln \left[ \lim_{n \to \infty} \left(1 + \frac{1}{n}\right)^n \right] \]
\tm $\lim$ out of $\ln$ (due to continuity).
  \[ 1 =  \lim_{n \to \infty}  \ln \left[  \left(1 + \frac{1}{n}\right)^n \right] \]
\tm $n$ out of $\ln$.
  \[ 1 =  \lim_{n \to \infty}  n \ln \left(1 + \frac{1}{n}\right)  \]
\tm $n = \frac{1}{r}$.
  \[ 1 =  \lim_{r \to 0}  \frac{1}{r} \ln (1 + r) \]
\tm sum $0 = \ln(1)$.
  \[ 1 =  \lim_{r \to 0}  \frac{ \ln (1 + r) + \ln(1)}{r} \]
\tm definition of derivative of $\ln$ calculated at $1$.
 \[ 1 =  \left. \ln'(x)\right\vert_{x=1} \]
 \[ 1 =  \left. \frac{1}{x}\right\vert_{x=1} \]
 \[ 1 = 1 \]
\ee

\hfill $\qed$
\end{frame}
%----------------------------------------------------------------------------------%


\section{About functions}


%----------------------------------------------------------------------------------%
\begin{frame}{Taylor expansion}{also known as polynomial expansion}



\defn 
\[  f(x) = \sum_{n=0}^{\infty} \frac{f\ \!^{(n)}(0) x^n}{n!}   \]

\pause 

\ 

\ 

Expanding...
\[ f(x) = \pause f(0) + \pause f'(0)x + \pause \frac{f''(0)x^2}{2!} + \pause \frac{f'''(0)x^3}{3!} + \ldots \]
\end{frame}
%----------------------------------------------------------------------------------%



\section{About the World Wide Web}


%----------------------------------------------------------------------------------%
\begin{frame}{Some important websites}
There are very interesting places on the web
\bi
\tm the 1st one is the most famous search engine \cite{1}.
\tm the 2nd one is a remarkable online encyclopedia \cite{2}.
\ei
\end{frame}
%----------------------------------------------------------------------------------%



%------------------------------------------------
\begin{frame}[fragile]{Some Java code}{}

\begin{lstlisting}
public class MainClass {
  public static void main(String[] args) {
    int limit = 20;
    int sum = 0;
    int i = 1;

    while (i <= limit) {
      sum += i++;
    }
    System.out.println("sum = " + sum);
  }
}

\end{lstlisting}

\end{frame}
%------------------------------------------------


%----------------------------------------------------------------------------------%
\begin{frame}{References}
\begin{thebibliography}{XX}
\bibitem[Google]{1} {\emph{Google:} \url{http://www.google.com}}
\bibitem[Wikipedia]{2} {\emph{Wikipedia:} \url{http://www.wikipedia.com}}
\end{thebibliography}
\end{frame}
%----------------------------------------------------------------------------------%



\titlepageINF


\end{document}